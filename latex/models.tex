\documentclass[a4paper,14pt,english]{extreport}

\usepackage{extsizes}
\usepackage{cmap}
\usepackage[T2A]{fontenc}
\usepackage[utf8]{inputenc}
\usepackage[english]{babel}

\usepackage{amsmath}
\usepackage{amsfonts}
\usepackage{amssymb}
\usepackage{graphicx}
\usepackage{mathtools}
\usepackage{makecell}
\usepackage{multirow}
\usepackage{booktabs}
\usepackage{commath}
\usepackage{longtable}
\usepackage{datetime2}

\usepackage[usenames, dvipsnames]{color}
\definecolor{fontColor}{RGB}{169, 183, 198}
\definecolor{pageColor}{RGB}{43, 43, 43}

\makeatletter
\let\mytagform@=\tagform@
\def\tagform@#1{\maketag@@@{\color{fontColor}(#1)}}
\makeatother



\newcommand\fracn[2]{\mathchoice
	{{\textstyle\frac{\,\scriptstyle#1}{\!\!\scriptstyle#2}}}
	{\frac{\,#1}{\!\!#2}}
	{\frac{\,#1}{\!\!#2}}
	{\frac{\,#1}{\!\!#2}}
}

\newcommand\fraceq[2]{\mathchoice
	{{\displaystyle\frac{\,\textstyle#1}{\!\!\textstyle#2}}}
	{\frac{\,#1}{\!\!#2}}
	{\frac{\,#1}{\!\!#2}}
	{\frac{\,#1}{\!\!#2}}
}

\usepackage[width=1\textwidth, font={color=fontColor}]{caption}

\renewcommand\theequation{{\color{fontColor}\arabic{equation}}}
\renewcommand\thetable{{\color{fontColor}\arabic{table}}}
\renewcommand{\thepage}{{\color{fontColor}\arabic{page}}}


\usepackage[pdftex,unicode,colorlinks = true,linkcolor = fontColor]{hyperref}

\renewcommand{\thesection}{\arabic{section}}

\author{Aleksandr Zhuravlyov \and Zakhar Lanetc}

\title{CFD Boltzmann Method\\Theory}

\date{\DTMnow}

\begin{document}

    \pagecolor{pageColor}
    \color{fontColor}
    %\maketitle
    %\newpage
    %\tableofcontents{\thispagestyle{empty}}
    %\newpage

    \section*{FVM Diffusion transient flow}
    
The employed model of diffusion is derived by substituting the first Fick's law into the continuity equation. In integral form, this model can be represented as follows:

    \begin{eqnarray}
        \label{eq:conductivity_integral}
        \int \limits_{V} \frac{\partial C}{\partial t} d V - \oint \limits_{S} D \vec{\nabla}C \; \vec{dS} = 0,
    \end{eqnarray}
    where $C = C\left(\vec{x}, t\right)$~--~concentration, $\vec{x}$ ~--~radius vectors, $t$~--~time, $V$~--~volume, $D$~--~diffusion coefficient~(diffusivity), $S$~--~surface area.
    
Equation (\ref{eq:conductivity_integral}) describes the diffusive flow in general. The relevant boundary and initial conditions should be chosen in order to specialize this equation for the research purposes. Thus, the following boundary conditions have been applied:
    
    \begin{eqnarray}
        \label{eq:conductivity_init}
        C\left(\vec{x}, \mathit{0}\right) = \hat{C}\left(\vec{x}\right), \; 
     \end{eqnarray}
    \begin{eqnarray}
      \label{eq:conductivity_bound}
       C\left(\vec{x}, t\right)\Big|_{\mathit{\Gamma}_D} \!\!= \tilde{C}\left(\vec{x}, t\right), \; D \vec{\nabla}C \left(\vec{x}, t\right)\Big|_{\mathit{\Gamma}_N} \!\!= \vec{G}\left(\vec{x}, t\right),
    \end{eqnarray}
where $\mathit{\Gamma} = \mathit{\Gamma_D} + \mathit{\Gamma_N}$ is the boundary region.

The finite-volume representation of Eqs. (\ref{eq:conductivity_integral}), (\ref{eq:conductivity_init}), and (\ref{eq:conductivity_bound}) can be described as follows:
    \begin{eqnarray}
        \label{eq:conductivity_num}
        \alpha \Delta_{t} - \sum_{\Delta S} \beta\Delta_{x}= 0,
    \end{eqnarray}
    \begin{eqnarray}
         \label{eq:eq:conductivity_init_num}
        C = \hat{C}, \; 
    \end{eqnarray}
    \begin{eqnarray}
    \label{eq:eq:conductivity_bound_num}
      \label{eq:conductivity_bound_num}
    C \Big|_{\mathit{\Gamma}_D}= \tilde{C}, \; \beta\Delta_x \Big|_{\mathit{\Gamma}_N} = G\Delta S,
    \end{eqnarray}

Parameters $\alpha$ and $\beta$ are expressed as follows:
    \begin{eqnarray}
        \label{eq:alpha_beta}
        \alpha = \frac{\Delta V}{\Delta t}, \;
        \beta= \frac{\overline{D} \Delta S}{\Delta L},
    \end{eqnarray}
where $\Delta V$~--~volume of a particular grid-block, $\Delta t$~--~numerical time step, $\overline{D}$~--~average diffusivity, $\Delta L$~--~distance between centres of two neighbouring grid-blocks whit respect to a particular surface, $\Delta S$~--~surface bounding a grid-block (positive or negative sign is defined by the normal direction).

The parameters $\Delta^{t}$ and $\Delta^{m}$ described by Eq. (\ref{eq:conductivity_num}) are expressed as follows:
    \begin{eqnarray}
    \label{eq:delta_num}
    \Delta_t = C^{n+1} - C^{n}, \Delta_x = \begin{cases}
    C_{+}^{n+\mathit1} - C_{-}^{n+\mathit1} : &\text{implicit},\\
        C_{+}^{n} - C_{-}^{n} : &\text{explicit},
    \end{cases}
    \end{eqnarray}
where subscript signs $+$ and $-$ indicate the position of the finite-volume block relative to the normal direction of the current surface element $\Delta S$.

\end{document}
